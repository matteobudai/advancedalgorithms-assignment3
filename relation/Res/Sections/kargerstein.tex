\section{Karger and Stein's Randomized Algorithm}\label{karger}

\begin{lstlisting}[mathescape=true]
	KARGER (G,k):
	A = +$\infty$
	for i = 1 to k:		
		t = RECURSIVE_CONTRACT(G)		
			if t < min:
					min = t
	return min
	
	RECURSIVE_CONTRACT(G=(D,W)):
	n= number of vertices in G
	if n<=6:
		Gp= CONTRACT(G,2)
		return weight of the only edge (u,v) in Gp
	t = n/$\sqrt{2}$+1
	for i = 1 to 2:
		Gi = CONTRACT(G,t)
		wi = RECURSIVE_CONTRACT(Gi)
	return min(w1,w2)
	
	CONTRACT(G=(D,W),k):
	n= number of vertices in G
	for i = 1 to n-k:
		(u,v) = EDGE_SELECT(D,W)
		CONTRACT_EDGE(u,v)
	return D,W
	
	CONTRACT_EDGE(u,v):
	D[u] = D[u]+D[v]-2W[u,v]
	D[v] = 0
	W[u,v] = W[v,u] = 0
	for each vertex w $\in$ V: except u and v:
		W[u,v] = W[u,w] + W[v,w]
		W[w,u] = W[w,u] + W[w,v]
		W[v,w] = W[w,v] = 0
		
	EDGE_SELECT(D,W)
	1. Choose u with probability proportional to D[u]
	2. Once u is fixed, choose v with probability proportional to W[u,v]
	3. return the edge (u,v)
	
\end{lstlisting}

This is a randomized algorithm for the computation of a graph.
In the next subsections we explain how we have implemented the data structure and the functions of the algorithm.

\pagebreak

\subsection{Data Structure}
For the implementation of the algortihm we used:
\begin{itemize}
	\item  \textbf{k}: is a constant ($log^2n$) used by Karger to repeat Recursive-Contract k times and to obtain an error with probability less or equal to 1/n;
	\item  \textbf{V}: is the list of nodes;
	\item  \textbf{D}: is the list of the sum of the weights of each node;
	\item  \textbf{W}: is the list of the graph with 3 parameters(node,node,weight). Each node is connected with the others and if they are not connected the third parameter is set to 0 otherwise is set to the correct weight.
\end{itemize}


\subsection{Implementation}
For the implementation of the algorithm we used these functions:
\begin{itemize}
	\item  \textbf{Karger(G,k)}: 
		\begin{enumerate}
		\item  This is the main function of the algorithm where we set the timeout to 60 to limit the execution time of large instances;
		\item  We start the time and we set the minimum cut to infinite;
		\item  We iterate k times to obtain an error with probability less or equal to 1/n;
		\item  If the time minus the starting time is greater than the timeout we break;
		\item  We execute a copy of V,W and D and then we call the function Recursive-Contract;
		\item  If we found a value less than our minimum we update it  and we set the discovery time;
		\item  In the end we print the Minimum Cut, the Total time and the discovery time.
		\end{enumerate}
	\item  \textbf{Recursive-Contract(V, W, D)}: In this function we follow exactly the function above with our data structure;
	\item  \textbf{Contract(s, V,  W, D)}: Also in this function we follow the function above. The only difference is that when we select and contract an edge (u,v) then we remove from V the vertex V to respect the contraction;
	\item  \textbf{Contract-Edge(u,v, W, D)}: Also in this case we follow the function above (We update D and W with the new values to execute the contraction of the edge) with our data structure;
	\item  \textbf{Edge-Select(V1, D, W)}: that used:
	\begin{enumerate}
		\item  \textbf{Random-Select(C)}: ;
		\item  \textbf{binarySearch(array, x)}: .
	\end{enumerate}
\end{itemize}



\subsection{Complexity}



\pagebreak